\documentclass{article}

\usepackage[utf8]{inputenc}
\usepackage[T1]{fontenc}
\usepackage[french]{babel}

\usepackage{fullpage}
\usepackage{listings}

\title{IFT2035 -- Concepts de languages de programmation \\ Devoir 2}
\author{Alex Élie \& Guillaume Poirier-Morency (p1053380)}

\begin{document}

  \maketitle

  \abstract

  \section{Fonctionnement général du programme}

  Le programme a été découpé en trois partie:
  \begin{itemize}
    \item les fonctions pour traiter les listes ;
    \item les fonctions pour traiter les noeuds ;
    \item la boucle principale.
  \end{itemize}

  Il y a deux fonctions de traitement de liste qui se sont montrée utile dans la
  réalisation du travail: \textsf{split} et \textsf{take}. \textsf{split} sépare
  une liste en une liste de sous-listes séparée sur un symbole. \textsf{take}
  prend $n$ éléments d'une liste donnée. Ces deux fonctions permettent de
  traiter les entrées des utilisateurs lorsque les chaînes sont représentées
  sous forme de liste de caractères.

  Toutes les fonctions qui opèrent sur les noeuds reconstruisent récursivement
  leur descendance en respectant l'objectif de la fonction. La fonction
  \textsf{node-delete} par exemple reconstruit sa descendance en omettant le
  noeud qui doit être supprimé. La seule différence est la fonction
  \textsf{node-search} qui ne fait que trouver un noeud dans la descendence d'un
  noeud donné sans reconstruire l'arbre.

  La fonction d'insertion \textsf{node-insert} s'occupe aussi de la substitution
  dans le cas ou l'utilisateur désire assigner une nouvelle définition à un
  noeud.

  Les traitements des arbres binaires et le rebalancement splay ont été séparé
  afin d'éviter la répétition de code. Le processus de splay s'applique sur le
  résultat des fonctions de noeuds sous forme de composition dans la boucle
  \textsf{traiter}, alors nous avons une seule et unique implantation pour cette
  opération.

  Les fonctions de rotation \textsf{ziz}, \textsf{zag}, \textsf{zig-zig}, etc...
  effectue la rotation correspondante à partir d'un noeud donné de sorte
  qu'elles soient complètement réutilisable par les autres fonctions du
  programme.

  \section{Résolution des problèmes de programmation}

  \subsection{Analyse syntaxique}

  \subsection{Représentation des dictionnaires}

  \subsection{Affichage des réponses}

  \subsection{Traitement des erreurs}

  \section{Comparaison avec l'expérience de développement en C}

  La concaténation d'une liste de chaînes est simple à exprimer en Scheme avec
  un \textsf{fold-left} et un \textsf{string-append} sur une liste d'éléments.
  En C, il faut allouer suffisament de mémoire, vérifier si le bloc est bien
  alloué et utiliser \textsf{strcat} dans une boucle.

  Le traitement des arbres est assez lourd en Scheme.

  Toutes les fonctions de traitement de noeuds sont sous la forme itérative.
  \textsf{splaytree}, \textsf{}

  \section{}

\end{document}
