\documentclass{article}

\usepackage[utf8]{inputenc}
\usepackage[T1]{fontenc}
\usepackage[french]{babel}

\usepackage{fullpage}
\usepackage{listings}
\usepackage{xcolor}

% solarized pour la coloration du code!
\definecolor{base03}{HTML}{002B36}
\definecolor{base02}{HTML}{073642}
\definecolor{base01}{HTML}{586E75}
\definecolor{base00}{HTML}{657B83}
\definecolor{base0}{HTML}{839496}
\definecolor{base1}{HTML}{93A1A1}
\definecolor{base2}{HTML}{EEE8D5}
\definecolor{base3}{HTML}{FDF6E3}
\definecolor{yellow}{HTML}{B58900}
\definecolor{orange}{HTML}{CB4B16}
\definecolor{red}{HTML}{DC322F}
\definecolor{magenta}{HTML}{D33682}
\definecolor{violet}{HTML}{6C71C4}
\definecolor{blue}{HTML}{268BD2}
\definecolor{cyan}{HTML}{2AA198}
\definecolor{green}{HTML}{859900}

\lstset{
    basicstyle=\ttfamily,
    sensitive=true,
    backgroundcolor=\color{base3},
    keywordstyle=\color{cyan},
    commentstyle=\color{base1},
    stringstyle=\color{blue},
    numberstyle=\color{violet},
    breaklines=true,
    literate=
  {á}{{\'a}}1 {é}{{\'e}}1 {í}{{\'i}}1 {ó}{{\'o}}1 {ú}{{\'u}}1
  {Á}{{\'A}}1 {É}{{\'E}}1 {Í}{{\'I}}1 {Ó}{{\'O}}1 {Ú}{{\'U}}1
  {à}{{\`a}}1 {è}{{\`e}}1 {ì}{{\`i}}1 {ò}{{\`o}}1 {ù}{{\`u}}1
  {À}{{\`A}}1 {È}{{\'E}}1 {Ì}{{\`I}}1 {Ò}{{\`O}}1 {Ù}{{\`U}}1
  {ä}{{\"a}}1 {ë}{{\"e}}1 {ï}{{\"i}}1 {ö}{{\"o}}1 {ü}{{\"u}}1
  {Ä}{{\"A}}1 {Ë}{{\"E}}1 {Ï}{{\"I}}1 {Ö}{{\"O}}1 {Ü}{{\"U}}1
  {â}{{\^a}}1 {ê}{{\^e}}1 {î}{{\^i}}1 {ô}{{\^o}}1 {û}{{\^u}}1
  {Â}{{\^A}}1 {Ê}{{\^E}}1 {Î}{{\^I}}1 {Ô}{{\^O}}1 {Û}{{\^U}}1
  {œ}{{\oe}}1 {Œ}{{\OE}}1 {æ}{{\ae}}1 {Æ}{{\AE}}1 {ß}{{\ss}}1
  {ç}{{\c c}}1 {Ç}{{\c C}}1 {ø}{{\o}}1 {å}{{\r a}}1 {Å}{{\r A}}1
  {€}{{\EUR}}1 {£}{{\pounds}}1
}

\title{IFT2035 -- Concepts de languages de programmation \\ Devoir 2}
\author{Alex Élie (p1012971) \& Guillaume Poirier-Morency (p1053380)}

\begin{document}

  \maketitle

  \section{Fonctionnement général du programme}
  L'interaction commence par l'appel de la fonction \textsf{go} qui s'occupe de
  lire l'entrée utilisateur, d'appeler la fonction \textsf{traiter}, d'imprimer
  son résultat et de repasser le dictionnaire au prochain appel.

  Lorsqu'une requête est fini d'être traitée par \textsf{traiter} elle renvoie
  une paire contenant dans le champ \textsf{car} la réponse à imprimer sous
  forme de liste de caractères et le dictionnaire dans le champs \textsf{cdr}.

  Le programme a été découpé en trois partie:
  \begin{itemize}
    \item les fonctions pour traiter les listes ;
    \item les fonctions pour traiter les noeuds ;
    \item la boucle principale.
  \end{itemize}

  Il y a deux fonctions de traitement de liste qui se sont montrée utile dans la
  réalisation du travail: \textsf{split} et \textsf{take}. \textsf{split} sépare
  une liste en une liste de sous-listes séparée sur un symbole. \textsf{take}
  prend $n$ éléments d'une liste donnée. Ces deux fonctions permettent de
  traiter les entrées des utilisateurs lorsque les chaînes sont représentées
  sous forme de liste de caractères.

  Toutes les fonctions qui opèrent sur les noeuds reconstruisent récursivement
  leur descendance en respectant l'objectif de la fonction. La fonction
  \textsf{node-delete} par exemple reconstruit sa descendance en omettant le
  noeud qui doit être supprimé. La seule différence est la fonction
  \textsf{node-search} qui ne fait que trouver un noeud dans la descendence d'un
  noeud donné sans reconstruire l'arbre.

  La fonction d'insertion \textsf{node-insert} s'occupe aussi de la substitution
  dans le cas ou l'utilisateur désire assigner une nouvelle définition à un
  noeud.

  \lstinputlisting[firstline=316,lastline=339,language=Lisp]{tp2.scm}

  Les traitements des arbres binaires et le rebalancement splay ont été séparé
  afin d'éviter la répétition de code. Le processus de splay s'applique sur le
  résultat des fonctions de noeuds sous forme de composition dans la boucle
  \textsf{traiter}, alors nous avons une seule et unique implantation pour cette
  opération.

  Les fonctions de rotation \textsf{zig}, \textsf{zag}, \textsf{zig-zig}, etc...
  effectue la rotation correspondante à partir d'un noeud donné de sorte
  qu'elles soient complètement réutilisable par les autres fonctions du
  programme.

  \lstinputlisting[firstline=93,lastline=107,language=Lisp]{tp2.scm}

  \section{Résolution des problèmes de programmation}

  \subsection{Analyse syntaxique}
  Une requête est lue à partir de la ligne de commande et elle sera ensuite
  analysée par la fonction traiter-ligne qui prend la requête et le
  dictionnaire (l'arbre binaire) en argument. Ensuite, cette requête est
  convertie en liste pour effectuer l'analyse plus facilement et la fonction
  traiter est appelée.

  La fonction \textsf{member} est utilisée pour vérifier la présence et la
  position du caractère \textsf{=} dans l'entrée de l'utilisateur. La fonction
  \textsf{take} permet de prendre la partie qui précède le premier caractère
  \textsf{=} et le \textsf{(cdr (member expr))} nous donne la
  définition en excluant ce symbole.

  Si la partie de la définition est vide, on fait un appel à la fonction
  \textsf{node-delete} avec le terme que l'utilisateur a fournit.

  On sépare la définition avec la fonction \textsf{split} qui sépare une liste
  en sous-listes séparée sur un symbole donné (un \textsf{+} dans ce cas
  spécifique). On traite ensuite la liste de définition avec
  \textsf{node-build-definitions} qui s'occupe de produire une liste de
  définitions correspondant aux termes.

  On insère ensuite le résultat dans l'arbre avec la fonction
  \textsf{node-insert}. Si la clé se trouve déjà dans l'arbre, la fonction
  s'occupe de faire la substitution des définitions du noeud existant.

  \subsection{Représentation des dictionnaires}
  Le dictionnaire est un arbre ou chaque noeud est listes de la forme
  \textsf{(nœud-gauche terme définitions noeud-droit)}. Le champ
  \textsf{definitions} est une liste de chaînes de définitions. La définition
  globale n'est pas concaténée afin d'être réutilisée dans l'arbre.

  \lstinputlisting[firstline=82,lastline=86,language=Lisp]{tp2.scm}

  Le dictionnaire est passé à chaque traitement de requêtes et il se modifie pour
  former une plus grande liste ou se contracter lors des suppressions.
  Différentes opérations affecte la forme du dictionnaire comme l'ajout, la
  suppression, le splay du dictionnaire ou encore une simple recherche.

  La recherche est l'opération la plus simple, on peut simplement traverser le
  dictionnaire avec une fonction en forme itérative et retouner \textsf{\#f} si
  le nœud n'est pas trouver ou retourner le nœud (lc terme définition rc) auquel
  on ira chercher la définition qui sera affichée dans la console.

  L'ajout d'une définition qui n'est pas déjà présente est assez simple. On veut
  ajouter un nœud à la fin de l'arbre en reconstruisant celui-ci  pendant la
  descente et on retourne le nouvel arbre avec le nœud ajouté. Les opérations de
  comparaison sont faites en comparant les chaînes des termes pour savoir où
  continuer la recherche et lorsqu'on trouve un nœud vide, on insère. Ensuite,
  l'opération splay est appelée.

  Pour modifier une définition, on traverse l'arbre en le reconstruisant et
  lorsque le terme courant est trouvé, on reconstruit le nœud avec la nouvelle
  définition. La fonction splay est alors appelée.

  Pour l'opération de splay, c'est un peu plus complexe. Trois fonctions sont
  utilisées pour permette de propager le nœud à splayer jusqu'à la racine.
  Lorsqu'on descend dans l'arbre, la fonction node-splay se charge de vérifier
  les 2 prochains enfants pour connaître quel type d'opération effectuer
  (zig,zag,zig-zag etc...). Lorsque le nœud à splayé est trouvé, on effectue
  l'opération de splay qui concorde avec la position du nœud et on se retrouve
  avec un arbre où le nœud qu'on veut splayer se propage 2 étages plus haut. Une
  autre fonction se charge de vérifier si le nœud à splayer est bien rendu à la
  racine et dans le cas contraire recommence node-splay.

  \subsection{Affichage des réponses}
  Dépendemment de la requête, diverses réponses sont affichés. On peut soit
  afficher la définition d'un terme recherché ou encore terme inconnu si le
  terme n'est pas trouvé. Nous n'avions pas à nous occuper de l'affichage car la
  fonction go s'en charge. Nous devions par contre s'assurer que traiter-ligne
  retourne bien une paire (réponse dictionnaire). Si la requête ne requiert pas
  d'affichage, le nouveau dictionnaire est tout de même retourné, mais avec une
  réponse vide.

  \subsection{Traitement des erreurs}
  Comprarativement au programme en C, il n'y avait pas beaucoup d'erreurs à
  gérer, car les structures de données sont uniformes et le typage est
  dynamique.

  Si une recherche n'aboutit pas, elle fait simplement retourner terme inconnue.

  Il n'y a pas non plus de manipulation mémoire et donc beaucoup de choses sont
  géré par le compilateur sans qu'on est à y penser. On peut toutefois s'assurer
  que la donnée attendu est bien la bonne en utilisant des fonctions de
  comparaison sur les types.

  Le code a été testé à l'aide de tests unitaires afin de garantir le bon
  fonctionnement de chaque partie du programme. Étant donné qu'il n'y a pas
  d'effet de bord dans notre programme, on peut assembler des fonctions sans se
  craintes de comportements étranges.

  \section{Comparaison avec l'expérience de développement en C}

  Comparativement au développement en C, notre programme Scheme contient entre
  200 et 250 lignes de moins.

  La concaténation d'une liste de chaînes est simple à exprimer en Scheme avec
  un \textsf{fold-left} et un \textsf{string-append} sur une liste d'éléments.
  En C, il faut allouer suffisament de mémoire, vérifier si le bloc est bien
  alloué et utiliser \textsf{strcat} dans une boucle.

  Le traitement des arbres est assez lourd en Scheme et les opérations de splay
  et delete qui nécéssite de réordonner les noeuds sont complexes à exprimer,
  car on ne peut pas faire d'affectation. Il fallait bien réfléchir à comment la
  fonction allait reconstruire l'arbre en sortie.

  Les opérations de recherches et d'insertion était simples à exprimer, car elle
  se résumait à un parcours récursif de l'arbre. Scheme facilite beaucoup les
  traitements récursifs.

  Le découpage en fonctions réutilisable nous a permis de faire de la
  composition de fonction au lieu d'implanter l'opération de splay dans chaque
  fonction de traitement d'arbre.

  Les fonctions \textsf{node-search}, \textsf{node-insert} et \textsf{splaytree}
  sont sous la forme itérative. Les autres fonctions étaient difficile à
  exprimer sous cette forme.

\end{document}
